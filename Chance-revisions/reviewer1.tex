\documentclass[11pt]{article}

%\usepackage{url}
%\usepackage{hyperref}
%\urlstyle{rm} %roman style urls

\usepackage{tikz}
\usepackage{times}
\usepackage{geometry}
\usepackage{graphicx}
\usepackage{color}
\usepackage[export]{adjustbox}
\usepackage{amsmath}
\usepackage{amsthm}
\usepackage{url}
\usepackage{natbib}
\usepackage{hyperref}

\usepackage{geometry}
\geometry{margin=1in}

%\setlength{\parindent}{0cm}

\newtheorem{thm}{Theorem}

%\DeclareMathOperator{\var}{var}

%\newcommand{\fatdot}{\,\cdot\,}

\setlength{\oddsidemargin}{0.50truein}
\setlength{\evensidemargin}{0.50truein}
\setlength{\textwidth}{5.5truein}
\setlength{\topmargin}{0.25truein}
\setlength{\textheight}{8.5truein}
\setlength{\headsep}{0.35truein}
\setlength{\headheight}{0.0truein}
\setlength{\topskip}{10.0pt}
\pagestyle{empty}

\begin{document}

\footnotesize
%\color{lightblue}
\begin{tikzpicture}
  \node at (0,0) {\includegraphics[height=0.5in, keepaspectratio=true]{yale_logo}};
  \node at (8.3,-1.75cm) {\begin{tabular}{l}
       \sc Daniel J. Eck, PhD \\
       \it Postdoctoral Associate \\
       \it Department of Biostatistics \\
       Yale School of Public Health \\
       60 College Street \\
       New Haven, CT 06510 \\
       %fax: 203-785-6912 \\ 
       \url{daniel.eck@yale.edu} \\ 
       \url{http://campuspress.yale.edu/danieleck/}
       \end{tabular} };
\end{tikzpicture}

\color{black}
\normalsize 

\bigskip

\begin{center}
{\LARGE Reviewer 1 checklist}
\end{center}


\noindent{\bf Comment}:
Summary: The manuscript approaches this from a different angle than I have 
seen in the past.  The binomial application in Section 4 seems sound and is 
easy to understand.  The part about weighting and sensitivity analysis and 
generally why this might be the right thing to do, is less clear.  The 
manuscript has several interesting observations.  The critiques are 
interesting though opinionated.  I like the concept and it seems potentially 
appropriate for Chance.  The manuscript needs minor revisions and 
clarifications including possibly shortening or excluding some sections. \\

\noindent{\bf Response}: Thank you for your interest in this work.  The 
manuscript is shortened, Section 6.4 is removed, and clarity has been added 
to the weighting and sensitivity analysis.  The critiques are still present, 
though the tone is softened.  Additional justification for the weighting has 
been added. \\



Required revisions: \\

\noindent{\bf Comment}:
It should be clarified what demographic subgroups are included/excluded at 
each point in time (from Table 1).  The first paragraph in the data section 
does not discuss race/ethnicity at all, while later paragraphs seem to suggest 
but are not crystal clear on whether and what adjustments are being made.  
I found that discussion in general to be long-winded and not necessary. \\

\noindent{\bf Response}: Thank you for your feedback.  The first paragraph is 
now clearer on race/ethnicity.  The entire discussion has been shortened.  \\



\noindent{\bf Comment}:
The weighting and sensitivity analysis should likely have further 
justification as to why this is a reasonable thing to do.  Having a 
half-page appendix on the weighting regime seems inappropriate.  Probably 
incorporate that into the section on this part of the analysis. \\

\noindent{\bf Response}: The weighting material is now incorporated into the 
text.  Further justification as to why the weighting and sensitivity 
analysis is a reasonable thing to do has been added. \\



\noindent{\bf Comment}: I don’t see what Section 6.4 really adds over 
previous sections.  Also seems out of place. \\

\noindent{\bf Response}: Thank you for pointing this out, Section 6.4 
is now removed. \\



Recommendations / questions: \\

\noindent{\bf Comment}:
Page 4, Table 1 -- not sure how the cumulative population proportions are 
relevant.  I found myself more interested to know the noncumulative proportion 
over the timeframe. \\

\noindent{\bf Response}: We define baseball players from earlier eras to be 
those that started their MLB careers in the 1950 season or before.  
Therefore the cumulative proportion in row 8 of Table 1 is one of the most 
important quantities in the paper. \\



\noindent{\bf Comment}:
Why did the authors stop at Top 25?  It might be interesting to add top 100.  
Does the ``trend'' continue past the top 25? \\

\noindent{\bf Response}: We included top 10 and top 25 lists because those 
lists are easily digestible by fans and these lists are abundant online.  
I am interested in this query as well.  The trend continues and is worse for 
fWAR, bWAR, and Ranker.  The trend is still apparent for the ESPN rankings 
of the top 100 players, but it is dampened.  The chance of extreme event in 
the ESPN top 100 list is about 1 in 125, which is between chances 
corresponding to ESPN's top 10 and top 25 lists.
 

\end{document}

