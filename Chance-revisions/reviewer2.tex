\documentclass[11pt]{article}

%\usepackage{url}
%\usepackage{hyperref}
%\urlstyle{rm} %roman style urls

\usepackage{tikz}
\usepackage{times}
\usepackage{geometry}
\usepackage{graphicx}
\usepackage{color}
\usepackage[export]{adjustbox}
\usepackage{amsmath}
\usepackage{amsthm}
\usepackage{url}
\usepackage{natbib}
\usepackage{hyperref}

\usepackage{geometry}
\geometry{margin=1in}

%\setlength{\parindent}{0cm}

\newtheorem{thm}{Theorem}

%\DeclareMathOperator{\var}{var}

%\newcommand{\fatdot}{\,\cdot\,}

\setlength{\oddsidemargin}{0.50truein}
\setlength{\evensidemargin}{0.50truein}
\setlength{\textwidth}{5.5truein}
\setlength{\topmargin}{0.25truein}
\setlength{\textheight}{8.5truein}
\setlength{\headsep}{0.35truein}
\setlength{\headheight}{0.0truein}
\setlength{\topskip}{10.0pt}
\pagestyle{empty}

\begin{document}

\footnotesize
%\color{lightblue}
\begin{tikzpicture}
  \node at (0,0) {\includegraphics[height=0.5in, keepaspectratio=true]{yale_logo}};
  \node at (8.3,-1.75cm) {\begin{tabular}{l}
       \sc Daniel J. Eck, PhD \\
       \it Postdoctoral Associate \\
       \it Department of Biostatistics \\
       Yale School of Public Health \\
       60 College Street \\
       New Haven, CT 06510 \\
       %fax: 203-785-6912 \\ 
       \url{daniel.eck@yale.edu} \\ 
       \url{http://campuspress.yale.edu/danieleck/}
       \end{tabular} };
\end{tikzpicture}

\color{black}
\normalsize 

\bigskip

\begin{center}
{\LARGE Reviewer 2 checklist}
\end{center}


\noindent{\bf Comment}:
The article has less explicit statistics in it than most articles from 
Chance.  This doesn't affect its value in general but it seems a bit light. \\

\noindent{\bf Response}: 
I agree, the article does have less explicit statistics and that shouldn't 
affect its value.  %The argument presented is deep and it undermines other 
%approaches to this problem which use far more explicit statistics.
The article is now shortened by about 1/3, perhaps the reduction in length 
will lessen its light appearance. \\



\noindent{\bf Comment}:
I generally sympathize with the author's argument, but I'm not sure there's 
enough empirical evidence to support it.  Specifically: what weights should we 
give to the various eras?  The author gives five possibilities: straight 
population, w1, w2, w3, and w4.  There are many others that could 
be chosen, and it's a matter of intuition which one you think is correct. 

Speaking for myself, I think none of them.  Intuitively, with no evidence -- 
that baseball was played much more frequently, and was a much larger part of 
the social fabric, in the earlier years of the century.  I've heard of kids 
gathering at the sandlot to play pickup baseball games, but never football, 
basketball, or hockey (in the US). 

Baseball was a much, much bigger deal, compared to other sports, in earlier 
times, and that's not picked up by the author's metrics. 

Consider hockey.  The NHL has players from Russia, Sweden, Finland, the USA, 
and many other hockey-playing countries.  But almost 40\% of players are from 
Canada.  Of the top 10 players listed on Sportsnet, 4 are Canadian, again 40\%.  
Others are Russian, Swedish, and Slovenian. 

There are three Russians and two Swedes -- a ratio of 1.5:1 -- even though the 
population ratio is around 14:1.  The Canada:Russia ratio is 4:3, with a 
population ratio of about 1:4.

I theorize it's the culture.  Hockey is bigger in Canada than any sport in the 
USA.  Kids are encouraged to play, good players are noted as exceptional in 
early childhood, and parents dream of their kids making the NHL.

Perhaps you could explain that with weights based on fan interest -- maybe 
Canada does have five times the fan interest of Russia.  And maybe Sweden 
does have twice the fan interest of Canada. 

But I doubt the numbers work out that nicely... Even after factoring in that 
there might be players still in Sweden or Russia that would make top 10 here.  

So I don't think there's enough empirical evidence.  I would ask that the 
author come up with a weighting that accurately predicts countries today.  
Then, treat previous eras as countries, and use the same weighting.  Then, 
the results might be strong enough to show something empirically. \\



\noindent{\bf Response}:  
Thank you for a well thought out critique of the sensitivity analysis.
The purpose of the weights is to best account for the changing popularity of 
baseball over time.  
The weighting regimes serve the role of a sensitivity analysis, these weights 
were designed to address, and in fact, overcompensate for any potential 
shortcomings of no weighting.
I now make this point clear in the paper.

I am not sure that these accounts of the popularity of hockey in Canada 
relative to other countries translate well to baseball.  The claim that 
hockey is much bigger in Canada than any sport in USA is certainly true and 
was probably always true.  
The following is taken from the article 
\url{https://www.justlanded.com/english/Canada/Canada-Guide/Culture/Ice-hockey-the-national-sport-of-Canada}: \\

\noindent ``In 2004 a poll was taken in Canada to find the 10 greatest 
Canadians of all time, millions of Canadians chose two hockey players within 
their list - Wayne Gretzky and Don Cherry.  Furthermore, having a hockey scene 
on the back of the Canadian five dollar note (see below) is just another 
example of how close to their hearts this sport really is. Some Canadians who 
feel strongly about the sport, believe that hockey has huge impacts on Canada, 
so much so that it defines it. There have been books written on the influence 
and connection between hockey and Canada, such as Jim Prime's book, How Hockey 
Explains Canada: The Sport That Defines a Country.'' \\

\begin{center}
\includegraphics[width=0.50\textwidth]{canada5dollarbill.jpeg}
\end{center}

I understand that this 2004 poll is not a random sample.  However, these 
results suggest that hockey in Canada is a much larger part of the culture 
than any particular sport is in the USA (a non random online poll for the 
greatest Americans of all time at Ranker lists its first sports figure 
at 43, see:  
\url{https://www.ranker.com/list/greatest-americans-ever/jacariah}).  

It may be the case that the weighting methods that I propose might not hold 
up in the hockey example.  It is hard to know for sure since historical 
polling data for the popularity of sports in countries other than the US is 
hard to find.  It may very well be the case that hockey has five times the 
fan interest in Canada than it has in Russia.  Surveys indicate that 
football is the preferred sport in Russia 
(\url{http://russia.com/activity/football/}).  I agree that 
Sweden probably does not have twice the fan interest of Canada today, 
although hockey is very popular in Sweden 
(\url{https://en.wikipedia.org/wiki/Ice_hockey_in_Sweden}).  
The all-time rankings of hockey players are much more favorable to 
Canadians and they fall in alignment with the narrative presented, 
the origins of the sport (\url{https://en.wikipedia.org/wiki/Ice_hockey}), 
and nostalgia (see
\url{https://www.thescore.com/nhl/news/1361102}, 
\url{https://seatgeek.com/tba/sports/the-top-10-best-nhl-players-of-all-time/}, 
\url{https://www.thetoptens.com/hockey-players/}, 
and 
\url{https://www.nhl.com/fans/nhl-centennial/100-greatest-nhl-players}).


In any event, I do not think that the hockey analogy is an indictment 
of the weights that I employ as a sensitivity approach within the context of 
baseball.  Polling data on the changing popularity of baseball in the US is 
easy to obtain and goes back as far as 1937.  It is reasonable to suggest 
that this information can serve as a useful proxy in determining the MLB 
eligible population.  The weighting regime w3 is motivated from the following 
graph which is taken from Gallup polling data at 
\url{https://news.gallup.com/poll/4735/sports.aspx}

\begin{center}
\includegraphics[width=0.75\textwidth]{Gallupfavoritesport.png}
\end{center}
%I can't say with certainty that this graphic captures the social fabric 
%component that baseball may have possessed in the earlier years of the 
%century.  However, 
This graphic is now added to the text.
%Intuitively, with no evidence -- 
%that baseball was played much more frequently, and was a much larger part of 
%the social fabric, in the earlier years of the century.  I've heard of kids 
%gathering at the sandlot to play pickup baseball games, but never football, 
%basketball, or hockey (in the US). 
The weighting regimes w1 and w2 are motivated from the Gallup article at 
\url{https://news.gallup.com/poll/6745/Baseball-Fan-Numbers-Steady-Decline-May-Pending.aspx?g_source=baseball%20interest&g_medium=search&g_campaign=tiles}
which notes that general interest in baseball has remained steady from 
1937 on.  This article is now added to the text.  This article and the above 
graphic suggests that it is a bit hyperbolic to state that baseball was 
played much more frequently, and was a much larger part of the social fabric, 
in the earlier years of the century.


I admit that the polling data only goes back to 1937, but I do not think that 
national interest in baseball was much different before 1937 than it was 
after.  For example as to why I think this, consider the 1927 Yankees which 
is one of the greatest baseball teams of all time, if not the greatest 
(\url{https://en.wikipedia.org/wiki/Murderers%27_Row}).  
However, the attendance 
figures for this great 1927 Yankees team were very modest in comparison to 
more modern Yankees seasons 
(\url{http://www.baseball-almanac.com/teams/yankatte.shtml})
even when taken relative to the population of New York City 
(\url{https://en.wikipedia.org/wiki/Demographic_history_of_New_York_City}).
Bleacher seat tickets at Yankee stadium cost anywhere from 50 to 75 cents  
in 1927 (\url{http://baseballguru.com/hfrommer/analysishfrommer80.html}) 
which is 7.10 to 10.65 in 2019 dollars. I do not think that cost is the 
explanation for modest attendance figures for the 1927 Yankees.  Also note 
that Yankees attendance is not limited by seating capacity 
(\url{http://www.baseball-almanac.com/stadium/yankee_stadium.shtml}).
Returning to hockey for a moment, in 1927 the Montreal Canadians had a much 
higher attendance than the Yankees relative to population 
(see \url{http://www.hockeydb.com/nhl-attendance/att_graph.php?tmi=6929} and 
\url{http://demographia.com/db-cancityhist.htm}).  The importance of hockey 
to Canadians is much greater than that of baseball to Americans. 


We are certainly taught that interest in baseball expanded in the roaring 
20s.  For example, see page 766 in the 8th grade history textbook 
\url{https://www.orange.k12.nj.us/cms/lib/NJ01000601/Centricity/Domain/434/United_States_History_Unit_8.pdf} 
used in Orange Public Schools in New Jersey.  
My freshman year history class taught the same thing, but I no longer 
have the textbook.
We do see an uptick in 
Yankees attendance starting in 1920 (this is the Babe Ruth season 
mentioned in the Introduction of the paper) which supports this narrative, 
but these figures are still modest in comparison to the attendance figures of 
more modern eras.  Additionally, there were no radio broadcasts of baseball 
games prior to 1920 
(\url{https://en.wikipedia.org/wiki/Major_League_Baseball_on_the_radio}).  
I do not think that interest in baseball could have been much larger before 
1920 considering that slugging Babe Ruth and the radio did not exist and 
attendance of baseball games was lower 
(\url{https://www.baseball-reference.com/leagues/MLB/misc.shtml}).  
Again, it is a bit hyperbolic to state that baseball was played much more 
frequently, and was a much larger part of the social fabric, 
in the earlier years of the century.

The anecdotal disappearance of sandlot baseball could partially be explained 
by the emergence of Little League baseball which started in 1939 and has 
greatly expanded ever since 
(\url{https://en.wikipedia.org/wiki/Little_League_Baseball}).

I think that the empirical evidence does support the use of the 
weighting regimes in the paper, especially in the manner in which they are 
used, and the interest in baseball compared to other sports in earlier times 
is picked up by these metrics.  
We also can clearly see that the weighting regimes were designed to address, 
and in fact, overcompensate for any potential shortcomings of no weighting.
%are chosen to be 
%antagonistic to the conclusions of the analysis with respect to the straight 
%population when reliable figures on baseball interest is missing.
 

\end{document}

