\documentclass[11pt]{article}\usepackage[]{graphicx}\usepackage[]{color}
%% maxwidth is the original width if it is less than linewidth
%% otherwise use linewidth (to make sure the graphics do not exceed the margin)
\makeatletter
\def\maxwidth{ %
  \ifdim\Gin@nat@width>\linewidth
    \linewidth
  \else
    \Gin@nat@width
  \fi
}
\makeatother

\definecolor{fgcolor}{rgb}{0.345, 0.345, 0.345}
\newcommand{\hlnum}[1]{\textcolor[rgb]{0.686,0.059,0.569}{#1}}%
\newcommand{\hlstr}[1]{\textcolor[rgb]{0.192,0.494,0.8}{#1}}%
\newcommand{\hlcom}[1]{\textcolor[rgb]{0.678,0.584,0.686}{\textit{#1}}}%
\newcommand{\hlopt}[1]{\textcolor[rgb]{0,0,0}{#1}}%
\newcommand{\hlstd}[1]{\textcolor[rgb]{0.345,0.345,0.345}{#1}}%
\newcommand{\hlkwa}[1]{\textcolor[rgb]{0.161,0.373,0.58}{\textbf{#1}}}%
\newcommand{\hlkwb}[1]{\textcolor[rgb]{0.69,0.353,0.396}{#1}}%
\newcommand{\hlkwc}[1]{\textcolor[rgb]{0.333,0.667,0.333}{#1}}%
\newcommand{\hlkwd}[1]{\textcolor[rgb]{0.737,0.353,0.396}{\textbf{#1}}}%
\let\hlipl\hlkwb

\usepackage{framed}
\makeatletter
\newenvironment{kframe}{%
 \def\at@end@of@kframe{}%
 \ifinner\ifhmode%
  \def\at@end@of@kframe{\end{minipage}}%
  \begin{minipage}{\columnwidth}%
 \fi\fi%
 \def\FrameCommand##1{\hskip\@totalleftmargin \hskip-\fboxsep
 \colorbox{shadecolor}{##1}\hskip-\fboxsep
     % There is no \\@totalrightmargin, so:
     \hskip-\linewidth \hskip-\@totalleftmargin \hskip\columnwidth}%
 \MakeFramed {\advance\hsize-\width
   \@totalleftmargin\z@ \linewidth\hsize
   \@setminipage}}%
 {\par\unskip\endMakeFramed%
 \at@end@of@kframe}
\makeatother

\definecolor{shadecolor}{rgb}{.97, .97, .97}
\definecolor{messagecolor}{rgb}{0, 0, 0}
\definecolor{warningcolor}{rgb}{1, 0, 1}
\definecolor{errorcolor}{rgb}{1, 0, 0}
\newenvironment{knitrout}{}{} % an empty environment to be redefined in TeX

\usepackage{alltt}
\usepackage[utf8]{inputenc}
\usepackage{amsmath}
\usepackage{amsfonts}
\usepackage{amscd}
\usepackage{amssymb}
\usepackage{natbib}
%\usepackage{fullpage}
\usepackage{url}
\usepackage{graphicx,times}
\usepackage{setspace}
\usepackage{ragged2e}

\usepackage{geometry}
\geometry{margin=1in}
\IfFileExists{upquote.sty}{\usepackage{upquote}}{}
\begin{document}

\begin{center}

\noindent {\huge \bf Challenging nostalgia and performance metrics in baseball} \\

\vspace{.2in}

Daniel J. Eck

\vspace{.01in}

\textit{Department of Biostatistics, Yale University, New Haven, CT, 06510. \\ \url{daniel.eck@yale.edu}}

\vspace{.1in}

\end{center}

%\begin{abstract}
%We show that the great baseball players that started their 
%careers before 1950 are overrepresented among rankings of baseball's all time 
%greatest players.  The year 1950 coincides with the decennial US Census that 
%is closest to when Major League Baseball (MLB) was integrated in 1947. 
%We also show that performance metrics used to compare players have substantial 
%era biases that favor players who started their careers before 1950. 
%In showing that the these players are overrepresented, no individual 
%statistics or era adjusted metrics are used.  Instead, we argue that the eras 
%in which players played are fundamentally different and are not comparable.  
%In particular, there were significantly fewer eligible MLB players available 
%at and before 1950.  As a consequence of this and other differences across 
%eras, we argue that popular opinion, performance metrics, and expert opinion 
%over include players that started their careers before 1950 in their rankings 
%of baseball's all time greatest players. 
%\end{abstract}

%\noindent{\bf Keywords}: Sabermetrics, binomial distribution, population dynamics, baseball ranking, era confounding \\


\section{Introduction}
\doublespacing
It is easy to be blown away by the accomplishments of great old time 
baseball players when you look at their raw or advanced baseball statistics.  
%% Consider making a chart for the mind-boggling seasons and careers.
These players produced mind-boggling numbers. For example, see 
Babe Ruth's batting average and pitching numbers, 
Honus Wagner's 1900 season, 
Ty Cobb's 1911 season, 
Walter Johnson's 1913 season, 
Tris Speaker's 1916 season, 
Rogers Hornsby's 1925 season, 
and
Lou Gehrig's 1931 season.
The statistical feats achieved by these players (and others) far surpass 
the statistics that recent and current players produce.  At first glance 
it seems that players from the old eras were vastly superior to the 
players in more modern eras, but is this true? 
Were the old timers actually better? 
In this paper, we investigate whether baseball players from earlier 
eras of professional baseball are overrepresented among the game's all-time 
greatest players according to popular opinion, performance metrics, and expert 
opinion.  We define baseball players from earlier eras to be those that 
started their MLB careers in the 1950 season or before.  
We chose this year because it coincides with the decennial US Census 
and is close to 1947, the year in which baseball became integrated. 

In this paper we do not compare baseball players via their statistical 
accomplishments.  Such measures exhibit era biases that are confounded with 
actual performance.  Consider the single season homerun record as an example. 
Before Babe Ruth, the single season homerun record was 27 by Ned Williams in 
1884 %\citep{reichler1985encyclopedia}.  
Babe Ruth broke this record in 1919 
when he hit 29 homeruns.  He subsequently destroyed his own record in 
the following 1920 season when he hit 54 homeruns.  The runner up in 1920 
finished the season with a grand total of 15 homeruns.  At this point in time 
homerun hitting was not an integral part of a batter's approach. %\citep{bref}.  
This has changed. Now, we often see multiple batters reach at least 30-40 
homeruns within one season and a 50 homerun season is not a rare 
occurrence. %\citep{bref, fangraphs}. 
In the 1920s, Babe Ruth stood head and shoulders above his peers due to a 
combination of his innate talent and circumstance.  
His approach was quickly emulated and became widely adopted. %\citep{bref}.  
However, Ruth's accomplishments as a homerun hitter would not stand out nearly 
as much if he played today and put up similar homerun totals.    
The example of homeruns hit by Babe Ruth and the impact they had relative 
to his peers represents a case where adjustment towards a peer-derived 
baseline fails across eras.  No one reasonably expects 1920 Babe Ruth to hit 
more than three times the amount of homeruns hit by the second best homerun 
hitter if the 1920 version of Babe Ruth played today.  
This is far from an isolated case.  

There are several statistical approaches used to compare baseball players 
across eras. 
Examples include 
wins above replacement as calculated by baseball reference (bWAR), %\citep{bref}, 
wins above replacement as calculated by fangraphs (fWAR), %\citep{fangraphs}, 
adjusted OPS+, %\citep{bref, fangraphs},
adjusted ERA+, %\citep{bref, fangraphs},
era-adjusted detrending \citep{petersen}, 
computing normal scores as in Jim Albert's work on a Baseball Statistics Course 
in the Journal of Statistics Education, 
and era bridging \citep{berry1999eras}. 
A number of these are touted to be season adjusted and the remainder are 
widely understood to have the same effect.  
In one way or another all of these statistical approaches compare the  
accomplishments of players within one season to a baseline that 
is computed from statistical data within that same season.  
This method of player comparison ignores talent discrepancies that exist 
across seasons as noted by Stephen J. Gould in numerous lectures and papers.
%\citet{gould1996fullhouse, schmidt2005concentration}.  
Currently, there is no definitive quantitative or qualitative basis for 
comparing these baselines, which are used to form intra-season player 
comparisons, across seasons.  These methods therefore fail to properly 
compare players across eras of baseball despite the claim that they are 
season adjusted.  

Worse still is that these approaches exhibit a favorable bias towards baseball 
players who played in earlier seasons 
\citep{schmidt2005concentration}.  
We explore this bias from two separate theoretical perspectives underlying how 
baseball players from different eras would actually compete against each 
other.  The first perspective is that players would teleport across eras to 
compete against each other.  From this perspective, the players from earlier eras 
are at a competitive disadvantage because, on average, baseball players have 
gotten better as time has progressed.
%\citep{gould1996fullhouse, schmidt2005concentration}.  
Specifically, it is widely acknowledged that 
fastball velocity, pitch repertoire, training methods, and management 
strategies have all improved over time.  
%\citep{lewis2004moneyball, keh2013offseason, thorn2014pitching, 
%doran2015velocity, arthur2016bullpen, castrovince2016velocity}.  
We do not find the teleportation perspective to be of 
much interest for these reasons.  The second perspective is that a player from 
one era could adapt naturally to the game conditions of another era if they 
grew up in that time. 
This line of thinking is challenging to current statistical methodology because 
adjustment to a peer-derived baseline no longer makes sense. 

Even in light of these challenges with the second perspective, we find that the 
players from earlier eras are overrepresented among baseball's all time greats.  
We justify our findings through the consideration of population dynamics which 
have changed drastically over time.  %\citep{gould1996fullhouse}. \\

%The only statistics that are necessary for this analysis are the year when a 
%player began his career and the eligible MLB population at that time. 
%The population data (in Section 2) is constructed and 
%used to formally test (in Section 4) whether or not too many old time players 
%are represented in ESPN's top 25 list \citep*{ESPN}. We do discover that there 
%are far too many old time players included in ESPN's top 25 list. The extent 
%of our conclusions conflicts with many conventional baseball evaluation 
%metrics as well as a sophisticated era-adjustment detrending metric 
%\citep*{petersen}. The reference \citet*{petersen} will henceforth be called 
%PPS due to repeated use. A detailed comparison and critique of the 
%methodologies underlying the evaluation metrics and our initial analysis is 
%then given. 
%kahrl2016wagner


\section{Data}











The eligible MLB population is not well-defined.  As a proxy, we can say 
that the eligible MLB population is the decennial count of males aged 
20-29 that are living in the United States (collected on years 1880, 1890,...) 
and Canada (collected on years 1881, 1891,...).
This information is readily obtainable and does not explicitly double count 
individuals over the course of its collection.  
The MLB eligible population is displayed in Table 1.
The cumulative proportion means that at each era, the population of the 
previous eras is also included. As an example of how to interpret this 
dataset, consider the year 1950. There were 11.59 
million males aged 20-29. The proportion of the historical eligible MLB 
population that existed at or before 1950 is 0.187. 


We now explain the specifics of MLB eligible population data recorded in 
Table 1. 
The MLB started in 1876 so our data collection begins with the 
1880 Census. Baseball was finally integrated in 1947.  As a result, 
African American and Hispanic demographic data is added to our dataset 
starting in 1960. The year 1960 is chosen because the integration of the MLB 
was slow as noted in Armour's work on the integration of baseball in the 
Society for American Baseball Research. %\citep{armour2016integration}. 
We obtain this demographic data for the United States 
eligible MLB population from the US Census for years 1880-1950 %\citep{census} 
and from Statistics Canada for years 1881-1951. %\citep{statcan}.
From 1960-2010, we use the United Nations Census %\citep{UN} 
for the United States eligible MLB population.  The latter census is also used 
to estimate the population in the global talent pool from which the MLB draws.  
We note that the black population in Canada is relatively negligible for time 
periods before 1960 as noted in Milan and Tran's work in Canadian Social 
Trends Spring. %\citep{milan2004blacks} 
Therefore these tallies are not included in the eligible MLB population 
before 1960.  The same is done for the Latin American population in Canada.  
%\bibitem[Milan and Tran(2004)Milan and Tran]{milan2004blacks}
%Milan, A. and Tran, K. (2004),
%\newblock ``Blacks in Canada: a long history,'' 
%\newblock {\em Canadian Social Trends Spring}, 2--7



% countries <- c("Mexico", "Aruba", "Panama", "Columbia", "Cuba", "Honduras", 
%  "Jamaica", 
%  "Bahamas", "Peru", "Dominican Republic", "Netherlands Antilles", 
%  "Puerto Rico", "Venezuela (Bolivarian Republic of)", "Nicaragua", #"Italy", 
%  "United States Virgin Islands", "United States of America")

African American and Hispanic US citizens were not the only groups 
discriminated against by pre- (and post-) integration baseball. Players from 
Central and South American countries and the Caribbean islands were also 
discriminated against.  Citizens from those countries are added to the 
eligible MLB player pool in 1960. The population data for these countries are 
obtained from the UN.  %\citet*{UN}.  
The countries included in our dataset are Mexico, the Dominican Republic, 
Venezuela, Cuba, Panama, Puerto Rico, Netherlands Antilles, Aruba, Honduras, 
Jamaica, the Bahamas, Peru, Columbia, Nicaragua, and the United States Virgin 
Islands.  In the 2000s, the MLB and minors saw an influx of Asian baseball 
players from Japan, South Korea, Taiwan, and the Philippines. The populations 
from these countries are added to the eligible MLB player talent pool for 
those years. We obtain the Japanese, South Korean, and Philippines census data 
from the UN %\citet*{UN}.  
The census data from Taiwan is estimated from the CIA World Factbook, %\citep{CIA}, 
the same age stratification was not obtainable so we use the population of 
males ages 0-14 for the 2010 Taiwanese MLB eligible population and males ages 
15-24 for the 2000 Taiwanese MLB eligible population.  In 2010, the MLB 
established a national training center in Brazil %\citep{Brazil}, 
as noted in Lor{\'e}'s work on the popularity of baseball in Brazil in the 
Culture Trip.
%\bibitem[Lor{\'e}(2017)Lor{\'e}]{Brazil}
%Lor{\'e}, Michael (2017),
%\newblock ``Baseball is Gaining Popularity in Brazil,''
%\newblock Culture Trip. 
%\url{https://theculturetrip.com/south-america/brazil/articles/baseball-is-gaining-popularity-in-brazil/}
Therefore we have included the age 20-24 Brazilian male 
population obtained from the UN %\citet{UN} 
into our MLB eligible population.  The United Nations Census does not have 
information after 2010.  We therefore estimate that the 2015 MLB eligible 
population is half of the 2010 MLB eligible population.  We expect that 
this underestimates the actual 2015 MLB eligible population because we 
have observed a constant increase in the overall MLB eligible population as 
time increases. 




% latex table generated in R 3.4.4 by xtable 1.8-3 package
% Tue May 28 18:40:55 2019
\begin{table}[ht]
\centering
\begin{tabular}{lccc}
  \hline
 & year & population & cumulative population proportion \\ 
  \hline
1 & 1880 & 4.440 & 0.013 \\ 
  2 & 1890 & 5.010 & 0.027 \\ 
  3 & 1900 & 5.580 & 0.043 \\ 
  4 & 1910 & 8.560 & 0.068 \\ 
  5 & 1920 & 8.930 & 0.093 \\ 
  6 & 1930 & 9.920 & 0.122 \\ 
  7 & 1940 & 11.130 & 0.154 \\ 
  8 & 1950 & 11.590 & 0.187 \\ 
  9 & 1960 & 18.420 & 0.240 \\ 
  10 & 1970 & 24.490 & 0.310 \\ 
  11 & 1980 & 33.930 & 0.407 \\ 
  12 & 1990 & 37.460 & 0.515 \\ 
  13 & 2000 & 60.660 & 0.689 \\ 
  14 & 2010 & 72.270 & 0.896 \\ 
  15 & 2015 & 36.140 & 1.000 \\ 
   \hline
\end{tabular}
\caption{Eligible MLB population throughout the years. The first column 
    indicates the year, the second column indicates the estimated eligible MLB 
    population size (in millions), and the third column indicates the proportion 
    of the eligible MLB population in row x that was eligbile at or before row x.} 
\end{table}






\section{The greats}

At a quick glance of this population dataset, one can see how small the 
proportion of the pre-1950 eligible MLB population actually was. 
To determine which players are the all-time greatest players, we consult four 
lists which reflect popular opinion, performance metrics, and expert opinion 
that purport to determine the greatest players.  The first 
list is compiled by ranker, %\citet{ranker2018greatest}
which is constructed entirely from popular opinion as determined by up and 
down votes.  
The second and third lists rank players by highest career WAR as calculated 
by baseball reference and fangraphs, respectively. 
%(technically speaking, these lists are rankings according to a statistic which 
%has no voice in saying who is definitively the best player).  
The fourth list is a ranking from ESPN %\citep{ESPN} 
and is based on expert opinion and statistics. %\citep{ESPN2015methodology}.
%szymborski

The rankings for all four lists are given in Table~\ref{top25}.  
As an example of the information contained in Table~\ref{top25} consider 
the greatest players of all time according to ESPN  
displayed in the fourth column.  
We see that 5 players that started their careers before 1950 are in the top 10 
all time and 11 players that started their careers before 1950 are in the top 
25 all time.  When the eligible MLB population is considered, it appears that 
the players from the earlier eras are overrepresented in this particular list.  
All of the lists that we consider exhibit this same bias towards the older 
players.  In the next Section we provide statistical evidence supporting this 
position.  




\begin{table}[h!]
\begin{center}
\begin{tabular}{lllll}
\hline
rank & Ranker & bWAR & fWAR & ESPN \\
\hline
1  & {\bf Babe Ruth}         & {\bf Babe Ruth}      & {\bf Babe Ruth}      & {\bf Babe Ruth}      \\
2  & {\bf Ty Cobb}           & {\bf Cy Young}       & Barry Bonds          & Willie Mays          \\
3  & {\bf Lou Gehrig}        & {\bf Walter Johnson} & Willie Mays          & Barry Bonds          \\
4  & {\bf Ted Williams}      & Barry Bonds          & {\bf Ty Cobb}        & {\bf Ted Williams}   \\
5  & {\bf Stan Musial}       & Willie Mays          & {\bf Honus Wagner}   & Hank Aaron           \\
6  & Willie Mays             & {\bf Ty Cobb}        & Hank Aaron           & {\bf Ty Cobb}        \\
7  & Hank Aaron              & Hank Aaron           & Roger Clemens        & Roger Clemens        \\
8  & Mickey Mantle           & Roger Clemens        & {\bf Cy Young}       & {\bf Stan Musial}    \\
9  & {\bf Rogers Hornsby}    & {\bf Tris Speaker}   & {\bf Tris Speaker}   & Mickey Mantle        \\
10 & {\bf Honus Wagner}      & {\bf Honus Wagner}   & {\bf Ted Williams}   & {\bf Honus Wagner}   \\
11 & {\bf Cy Young}          & {\bf Stan Musial}    & {\bf Rogers Hornsby} & {\bf Lou Gehrig}     \\
12 & {\bf Walter Johnson}    & {\bf Rogers Hornsby} & {\bf Stan Musial}    & {\bf Walter Johnson} \\
13 & {\bf Joe Dimaggio}      & {\bf Eddie Collins}  & {\bf Eddie Collins}  & Greg Maddux          \\
14 & Sandy Koufax            & {\bf Ted Williams}   & {\bf Walter Johsnon} & Rickey Henderson     \\ 
15 & Ken Griffey Jr.         & {\bf Pete Alexander} & Greg Maddux          & {\bf Rogers Hornsby} \\
16 & {\bf Jimmie Foxx}       & Alex Rodriguez       & {\bf Lou Gehrig}     & Mike Schmidt         \\
17 & {\bf Tris Speaker}      & {\bf Kid Nichols}    & Alex Rodriguez       & {\bf Cy Young}       \\
18 & {\bf Joe Jackson}       & {\bf Lou Gehrig}     & Mickey Mantle        & Joe Morgan           \\
19 & Mike Schmidt            & Rickey Henderson     & Randy Johnson        & {\bf Joe Dimaggio}   \\
20 & Nolan Ryan              & Mickey Mantle        & {\bf Mel Ott}        & Frank Robinson       \\
21 & {\bf Christy Mathewson} & Tom Seaver           & Nolan Ryan           & Randy Johnson        \\
22 & Roberto Clemente        & {\bf Mel Ott}        & Mike Schmidt         & Tom Seaver           \\
23 & Albert Pujols           & {\bf Nap Lajoie}     & Rickey Henderson     & Alex Rodriguez       \\
24 & {\bf Cap Anson}         & Frank Robinson       & Frank Robinson       & {\bf Tris Speaker}   \\
25 & Greg Maddux             & Mike Schmidt         & Burt Blyleven        & Steve Carlton        \\
 & & & & \\
pre-1950 in top 10 &   7 \,/\, 10  &   6 \,/\, 10  &   6 \,/\, 10  &   5 \,/\, 10  \\
pre-1950 in top 25 &  15 \,/\, 25  &  15 \,/\, 25  &  12 \,/\, 25  &  11 \,/\, 25  \\
\hline
\end{tabular}
\end{center}
\caption{Lists of the top 25 greatest baseball players to ever play in the 
  MLB according to Ranker.com (1st column), bWAR (2nd column), 
  fWAR (3rd column), and ESPN (4th column). Players that started their career 
  before 1950 are indicated in bold. The last two rows count the number of players 
  that started their careers before 1950 in each of the top 10 and top 25 lists 
  respectively.}
\label{top25}
\end{table}





\section{Statistical evidence}
\label{sec:Stats}

We now provide evidence that there are too many players that started their 
careers before 1950 included in top 10 and top 25 lists displayed 
in Table~\ref{top25}.  
%In order to conduct these statistical tests, 
We require two assumptions for the validity of our calculations which we will 
explore in detail in the next Sections. 
These assumptions are: 
\begin{itemize}
\item First, we assume that innate talent is uniformly distributed over the 
  eligible MLB population over the different eras.
\item Second, we assume that the outside competition to the MLB available by 
  other sports leagues after 1950 is offset by the increased salary 
  incentives received by MLB players after 1950.
\end{itemize}

With these assumptions in mind we calculate the probability that at least x 
people from each top 10 and top 25 list in Table~\ref{top25} started their 
career before 1950 using the proportion depicted in Table 1.  Consider the 
bWAR list for example.  According to bWAR, we see that 6 of the top 10 
players started their career before 1950.  From Table 1 we see that the 
proportion of the MLB eligible population that played at or  
before 1950 was approximately 0.187.  
We then calculate the probability that one would expect to observe 6 or more 
individuals in a top 10 list from that time period where the chance of 
observing each individual is about 0.187.  We calculate 
this probability using the Binomial distribution.%, details of this calculation 
%are provided in Appendix A. 
We perform the same type of extreme event 
calculation for each top 10 and top 25 list depicted in Table~\ref{top25}.  
The results are provided in Table~\ref{probvalues}.








\begin{table}[h!]
\begin{center}
\begin{tabular}{lllll}
\hline
  &  Ranker  &  bWAR  &  fWAR  &  ESPN \\
  \hline
probability of extreme event in top 10 list 
  & 0.000562 
  & 0.00448 
  & 0.00448 
  & 0.0249 \\
probability of extreme event in top 25 list 
  & 0.0000057 
  & 0.0000057 
  & 0.000826 
  & 0.00322 \\
chance of extreme event in top 10 list 
  & 1 in 1780 
  & 1 in 223 
  & 1 in 223 
  & 1 in 40 \\
chance of extreme event in top 25 list 
  & 1 in 174816 
  & 1 in 174816 
  & 1 in 1210 
  & 1 in 310 \\
  \hline
\end{tabular}
\end{center}
\caption{The probability and chance (1 in 1/probability) of each extreme event 
  calculation corresponding to the four lists in Table~\ref{top25}.}
\label{probvalues}
\end{table}

As an example of how to interpret the results of Table~\ref{probvalues}, 
continue with bWAR's top 10 list.  Table~\ref{probvalues} shows that the 
probability of observing 6 or more players that started their career at 
or before 1950 of the top 10 all time players, based on population 
dynamics, is about 0.00448 
(a chance of 1 in 223).
The same interpretation applies to the other cells of Table~\ref{probvalues}.  
The results provided in Table~\ref{probvalues} present overwhelming evidence 
that players who started their careers before 1950 are overrepresented in top 
10 and top 25 lists from the perspectives of fans, analytic assessment of 
performance, and experts' rankings.  These findings are a testament to 1) how 
sparse the MLB eligible population was at and before 1950 relative to the MLB 
eligible population after 1950 and 2) how incorporating relevant population 
dynamics leads to completely different conclusions.




\section{Assumptions and Sensitivity Analysis}
\label{sec:Assumptions}

The probabilities and chances displayed in Table~\ref{probvalues} are valid 
under the two assumptions given in the previous Section.  In the first of 
these assumptions we specify that innate talent is evenly dispersed across 
eras.  We do not fully believe that this assumption holds because the 
distribution of innate talent has improved over time as the eligible MLB 
population has expanded as noted by Stephen J. Gould,  
Christina Kahrl at ESPN, and in 
Martin B. Schmidt and David J. Berri's work on concentration of baseball 
talent in the Journal of Sports Economics.
%\citep{gould1996fullhouse, schmidt2005concentration, kahrl2016wagner}.  
This suggests that the probabilities displayed in Table~\ref{probvalues} are 
conservative.  If we had refined data on the talent of those that strived 
to play professional baseball then the calculations in Table~\ref{probvalues} 
would be even more extreme.

The second assumption states that the pool of 
talent available in the MLB has not been diminished by other sports leagues 
because of increased salary incentives to play baseball.  We note that the 
National Basketball Association (NBA) and the National Football League (NFL) 
started in 1946 and 1920 respectively %\citep{NBA, NFL} 
with both sports greatly rising in popularity since the inception of their 
respective professional leagues.  That being said, it is widely known that 
the increases in MLB player salaries have been substantial.  
%\citep{badenhausen2016salary, haupert2016salary, shaikin2016salary, 
%radcliffe2018salary}.  
%As examples, the minimum MLB salary in 1967 was \$6,000 which is about 
%\$45,000 in 2018 dollars \citep{shaikin2016salary} and the highest salary 
%right before baseball was integrated in 1947 was \$70,000 which is about 
%\$800,000 in 2018 dollars \citep{haupert2016salary}.  In 2015, the 
%minimum and average MLB salaries are \$507,000 and \$3,952,252, 
%respectively, which are about \$530,000 and \$4,150,000, respectively, in 
%2018 dollars \citep{badenhausen2016salary}.  
For example, the 1967 census lists the median US household income as \$7,200. 
%The minimum MLB salary in 1967 was below the median US household income of 
%\$7,200 \citep{census1967} 
The minimum MLB salary at that time was \$6,000 as noted by the LA Times 
sports writer Bill Shaiken in a piece titled ``A look at how Major League 
Baseball salaries have grown by more than 20,000\% the last 50 years.''
%\$45,000 in 2018 dollars \citep{shaikin2016salary}
In 2015 the minimum MLB salary places earners around the top 1\% of 
US household income.
%The average MLB salary in 1920 was \$5,000 per year \citep{radcliffe2018salary}
%which placed earners of this income in only about the top 35\% of personal income 
%\citep{IRS1920}.  Today, the average salary places earners well above 
%the threshold of the top 1\% of earners \citep{radcliffe2018salary}.  
Brent Radcliff's article ``Baseball Greats Who Were Paid Like Benchwarmers''
has plenty more surprising MLB salary figures.
In summary, baseball players made far less than they do today relative to the 
general US population.  Furthermore, it is unclear that one could consider 
playing professional baseball to be a lucrative career in the older eras. 

These strikingly different salary figures offer evidence that while other 
professional leagues may have drawn from the eligible MLB talent pool, 
salary incentives have led to an increase in the overall quality of MLB 
players.  This suggests that our second assumption is conservative, 
again, if we had refined data on the eligible MLB population then the 
calculations in Table~\ref{probvalues} would be even more extreme.  











However, we are not convinced that this is the case with any certainty, 
it may be that our second assumption suffers some modest violations.  
To account for this possibility we consider a sensitivity 
analysis applied to the findings in Table~\ref{probvalues}.  For this 
sensitivity analysis we weight the decennial populations displayed in Table 1 
to reflect the overall interest that the US population has had in the game of 
baseball over time.  The four specific weighting schemes, denoted by w1, w2, 
w3, and w4, and their full descriptions are given in the Appendix.  We briefly 
describe these weighting schemes.  The first and second weighting schemes 
weigh populations with respect to overall interest in baseball.  
No information is given for pre-1940 baseball.  As a result of this 
we weigh years before 1940 with more weight than years after 1940 
and w2 places more weight on pre-1940 populations than w1.  
The third weighting scheme weighs populations based on favorite sport 
information.  No information is given for pre-1940 baseball.  As a result of 
this we give the highest weight observed at or after 1940 for all 
pre-1940 years.  The fourth weighting scheme is an average of w2 and w3.  
All of these weighting schemes suggest that the eligible MLB population was 
more interested in reaching the MLB in earlier eras than in modern eras.  








\begin{table}[h!]
\begin{center}
\begin{tabular}{llllll}
\hline
 weight & &  Ranker  &  bWAR  &  fWAR  &  ESPN \\
 \hline
w1 & probability of extreme event in top 10 list 
  & 0.00121 
  & 0.00839 
  & 0.00839 
  & 0.0406 \\
& probability of extreme event in top 25 list 
  & 0.0000267 
  & 0.0000267 
  & 0.0025 
  & 0.00845 \\
& chance of extreme event in top 10 list 
  & 1 in 824 
  & 1 in 119 
  & 1 in 119 
  & 1 in 25 \\
& chance of extreme event in top 25 list 
  & 1 in 37519 
  & 1 in 37519 
  & 1 in 401 
  & 1 in 118 \\
  & & & & & \\
w2 & probability of extreme event in top 10 list 
  & 0.0023 
  & 0.0141 
  & 0.0141 
  & 0.0604 \\
& probability of extreme event in top 25 list 
  & 0.0000944 
  & 0.0000944 
  & 0.00608 
  & 0.0182 \\
& chance of extreme event in top 10 list 
  & 1 in 434 
  & 1 in 71 
  & 1 in 71 
  & 1 in 17 \\
& chance of extreme event in top 25 list 
  & 1 in 10595 
  & 1 in 10595 
  & 1 in 164 
  & 1 in 55 \\
  & & & & & \\
w3 & probability of extreme event in top 10 list 
  & 0.0274 
  & 0.0989 
  & 0.0989 
  & 0.256 \\
& probability of extreme event in top 25 list 
  & 0.0102 
  & 0.0102 
  & 0.133 
  & 0.24 \\
& chance of extreme event in top 10 list 
  & 1 in 36 
  & 1 in 10 
  & 1 in 10 
  & 1 in 3.9 \\
& chance of extreme event in top 25 list 
  & 1 in 98 
  & 1 in 98 
  & 1 in 7.5 
  & 1 in 4.2 \\
  & & & & & \\  
w4 & probability of extreme event in top 10 list 
  & 0.00584 
  & 0.0296 
  & 0.0296 
  & 0.106 \\
& probability of extreme event in top 25 list 
  & 0.000575 
  & 0.000575 
  & 0.021 
  & 0.0524 \\
& chance of extreme event in top 10 list 
  & 1 in 171 
  & 1 in 34 
  & 1 in 34 
  & 1 in 9.4 \\
& chance of extreme event in top 25 list 
  & 1 in 1740 
  & 1 in 1740 
  & 1 in 48 
  & 1 in 19 \\
  \hline
\end{tabular}
\end{center}
\caption{The probability and chance (1 in 1/probability, rounded) 
  of each extreme event calculation corresponding to the four lists in 
  Table~\ref{top25} after the eligible MLB population in Table 1 is 
  weighted according to the four weighting schemes that are described in 
  the Appendix.}
\label{probvalues.weights}
\end{table}



The results of weighting populations and then recalculating 
the probabilities and chances displayed in Table~\ref{probvalues} 
are displayed in Table~\ref{probvalues.weights}.  The 
conclusions from weighting populations with respect to w1 and w2 in 
Table~\ref{probvalues.weights} are consistent with those in 
Table~\ref{probvalues}.  It is very unlikely that such a sparsely populated 
time period could have produced so many historically great baseball players.  
The third weighting scheme presents some conflicting conclusions.  When 
weighting populations with respect to w3 we see that popular opinion, bWAR, 
and fWAR overrepresent players who started their careers before 1950.  
However the same is not so for the ESPN lists and the fWAR lists, 
the probabilities depicted are low but are not that extreme.  
This gives skeptics to our approach some evidence in their favor.  
We will return to this finding later.  The conclusions from weighting 
populations with respect to w4 is largely in alignment with those of the 
original data, with a possible exception being the conclusion drawn from 
ESPN's top 10 list.  The overall finding of this sensitivity analysis is 
that weighting populations with respect to fan interest in baseball does 
not alter the overall conclusion of the original analysis.  It is very 
unlikely that such a sparsely populated time period could have produced 
so many historically great baseball players.  





\section{Critique of additional comparison methods}

In the previous Sections we observe that popular opinion (i.e., nostalgia) 
and performance metrics are in conflict with the population dynamics that 
underlie the distribution of the eligible MLB talent pool.  
%Our evaluation metric model free analysis of great players conflicts with the 
%top 25 list in Table~\ref{top25}, conventional evaluation metrics such as Wins 
%Above Replacement (WAR), and a more sophisticated era-adjusted detrending 
%metric of PPS. The underlying principles of these competing metrics are now 
%compared with out methodology. We start with WAR.
We now critique additional methods that compare players across eras.  



\subsection{Critique of versus your peers methods}
\label{WARcritique}

There are several methods which are used to compare players across eras that 
do so by computing a baseline achievement threshold within one season and then 
compare players to that baseline.  These methods then rank players by how far 
they stood above their peers, the greatest players were better than their peers 
by the largest amount. 
As noted in the Introduction, examples include 
wins above replacement as calculated by baseball reference (bWAR), 
wins above replacement as calculated by fangraphs (fWAR), 
adjusted OPS+,
adjusted ERA+,
and computing normal scores.
This manner of player comparison is purely statistical and it ignores talent 
discrepancies that exist across seasons.  We have shown that this is a 
fundamentally flawed methodological approach that can exhibit major biases in 
player comparisons as evidenced by career bWAR and fWAR.  Adjusted OPS+ is a 
worse offender than bWAR or fWAR and adjusted ERA+ is right in line with ESPN 
rankings. 

%Philosophically speaking, WAR is a one-number summary of the added value of a 
%player over a ``replacement" level player. It is a statistic but it is not 
%thought about as a statistic in the way statisticians think about statistics. 
%For instance, the replacement level player, which is central to the 
%interpretation of WAR, is not a fixed and known entity but rather a random 
%outcome. In each year, the perceived value of the replacement player changes 
%based on the individuals comprising the statistical outcome of that year's 
%baseball season. Therefore, the most central component to the calculation of 
%WAR is both random and is subject to a strong era-effect. The ESPN top 25 list 
%used a variation of WAR to compare players \citep*{szymborski}. Another 
%disadvantage of WAR is that it is more of a philosophical concept than a 
%calculable statistic. There are many different ways to calculate WAR 
%\citep*{bref, fangraphs} and these differing calculations assign different 
%values to the same performance. The arbitrariness of WAR as an actual 
%statistical tool coupled with its inherent randomness makes it a blunt tool 
%for comparing performances across eras. As an example of this, consider the 
%defensive comparison between Ozzie Smith's 1983 season and Rogers Hornsby's 
%1917 season provided in Table~\ref{Rajah}. Clearly Ozzie Smith had the 
%superior defensive season but Rogers Hornsby's defensive WAR is substantially 
%higher. In all, we see that it does not make sense to use WAR as a way to 
%compare players across eras.


%\begin{table}[h!]
%\begin{center}
%\begin{tabular}{l|cccccc}
%& pos & Fld\% & Chances & Errors & RF/9 & dWAR \\
%\hline
%Rogers Hornsby & SS & 0.939 & 847 & 52 & 5.62 & 3.5 \\
%Ozzie Smith    & SS & 0.975 & 844 & 21 & 5.51 & 2.3 \\
%\hline
%\end{tabular}
%\end{center}
%\caption{Comparison of Rogers Hornsby's 1917 defensive season and 
%  Ozzie Smith's 1983 defensive season.}
%\label{Rajah}
%\end{table}


\subsection{Critique of detrending}
In this Section we describe and critique the methodology of \citet{petersen} 
in the context of comparing players across eras.  We will refer to their 
paper as PPS due to repeated mentions.  
As described in PPS, they detrend player statistics by normalizing 
achievements to seasonal averages, which they claim accounts for changes in 
relative player ability resulting from both exogenous and endogenous factors, 
such as 
talent dilution from expansion, 
equipment and training improvements, 
as well as performance enhancing drug usage. 
As an argument in favor of players from earlier time periods, 
talent dilution from expansion is terribly misunderstood by PPS.  
If anything, the talent pool was more diluted in the earlier eras of 
baseball because of a small relative eligible population size and the 
exclusion of entire populations of people on racial grounds.  
See Table~\ref{dilution} for the specifics. As an argument in favor of old 
time players, equipment and training improvements is not without fault 
because the same improvements are available to every competitor as well.  
PPS also fails to account for the effect of modern day filming as a way to 
gain insight on opponents' strengths and weakness in their training 
improvements assumption.  PPS also fails to account for increases in salary 
compensation enjoyed by MLB players in more modern eras.  

\begin{table}[h!]
\begin{center}
\begin{tabular}{lrccc}
\hline
year & eligible pop. & number of teams & roster size & eligible pop. per roster spot \\
\hline
1880 & 4.44  & 8  & 15 & 37   \\
1890 & 5.01  & 8  & 15 & 41.7   \\
1900 & 5.58  & 8  & 15 & 46.5   \\
1910 & 8.56  & 16 & 25 & 21.4  \\
1920 & 8.93  & 16 & 25 & 22.3  \\
  %& & & &  \\
1930 & 9.92  & 16 & 25 & 24.8  \\
1940 & 11.13  & 16 & 25 & 27.8  \\
1950 & 11.59  & 16 & 25 & 29  \\
1960 & 18.42  & 16 & 25 & 46.1  \\
1970 & 24.49  & 24 & 25 & 40.8  \\
  %& & & &  \\
1980 & 33.93 & 26 & 25 & 52.2 \\
1990 & 37.46 & 26 & 25 & 57.6 \\
2000 & 60.66 & 30 & 25 & 80.9 \\
2010 & 72.27 & 30 & 25 & 96.4 \\
\hline
\end{tabular}
\end{center}
\caption{Relative talent dilution when considering the eligible MLB population 
  sizes at select time periods. Eligible population totals are in millions in 
  column 2 and are in thousands in column 5. }
\label{dilution}
\end{table}


The principle assumptions about baseball used as justification for PPS 
detrending are head scratchers at best.  The mathematics of PPS detrending 
is also questionable in the context of comparing baseball players across eras. 
%We now describe the mathematical formulation of the PPS detrending method.  
%They first calculate the prowess $P_i(t)$ of an individual player $i$ as 
%\begin{equation} \label{prowess}
%  P_i(t) = x_i(t)/y_i(t)
%\end{equation}
%where $x_i(t)$ is an individual's total number of successes out of his/her 
%total number of opportunities $y_i(t)$ in a given year $t$.  To compute the 
%league-wide average prowess, PPS then computes the weighted average for 
%season $t$ over all players
%\begin{equation} \label{avg-prowess}
%  \langle P(t) \rangle = \frac{\sum_i x_i(t)}{\sum_i y_i(t)} 
%    = \sum_i w_i(t)P_i(t)
%\end{equation}
%where
%\begin{equation} \label{weights}
%  w_i(t) = \frac{y_i(t)}{\sum_i y_i(t)}.
%\end{equation}
%The index $i$ runs over all players with at least $y^{\prime}$ opportunities 
%during year $t$, and $\sum_i y_i$ is the total number of opportunities of all 
%$N(t)$ players during year $t$.  A cutoff $y^{\prime} = 100$ is employed to 
%eliminate statistical fluctuations that arise from players with very short 
%seasons.  The PPS detrended metric for the accomplishment of player $i$ in 
%year $t$ is then given as,
%\begin{equation} \label{detrended}
%  x_i^D(t) = x_i(t)\frac{\overline{P}}{\langle P(t) \rangle}
%\end{equation}
%where $\overline{P}$ is the average of $\langle P(t) \rangle$ over the entire 
%period,
%$$ 
%  \overline{P} = \frac{1}{110} \sum_{t=1900}^{2009} \langle P(t) \rangle. 
%$$
%The PPS detrended metric $x_i^D(t)$ has its merits. It does adjust individual 
%prowess towards the historical average prowess.  It is also dimensionless in a 
%physical sense with respect to the measured accomplishment. However, this 
%metric is self-fulfilling; it is specifically constructed in alignment with 
%the opinion that prowesses $\langle P(t) \rangle$ are simple deviations about 
%$\overline{P}$ that are best addressed by shrinking era effects towards 
%$\overline{P}$.  Furthermore, PPS states that prowess is a nonstationary 
%time series which implies that the distribution of $\overline{P}$ can differ 
%wildly when computed across different time windows. 
PPS notes that the evolutionary nature of competition results in a 
non-stationary rate of success.  As already noted, PPS detrends player 
statistics by normalizing achievements to seasonal averages.  
The normalization goes as follows: 
	Suppose a player hits 40 homeruns in a given season and that the league 
	average prowess for homerun hitting in that season is 10 homeruns. If the 
	historical average prowess for homerun hitting is 5 homeruns then our 
	player's detrended homerun count in that particular season is 
	$40\times(5/10) = 20$.  In general, the detrending formula is 
	$Y \times (\text{historic prowess} / \text{league prowess})$ where $Y$ is 
  individual prowess for a particular player in a given season.
Fundamentally different approaches for detrending are advocated in 
authoritative textbooks such as 
	Introduction to Time Series and Forecasting,
		by
	Peter J. Brockwell and Richard A. Davis,
	Time Series Analysis and Its Applications With R Examples, 
	  by 
	  Robert H. Shumway and David S. Stoffer, 
	and
	Time Series Analysis and Forecasting by Example, 
	  by
	  S{\o}ren Bisgaard and Murat Kulahci,
and all of the authors conversations with statisticians.  
%None of which involve the 
%incorporation of an implicitly volatile quantity. 

%The supposition that $x_i^D(t)$ offers an objective comparison of players 
%across time is questionable for reasons previously mentioned.  
%The average $\overline{P}$ treats all eras as equals in its computation. 
%Different eras should be given different weights reflecting richness of 
%the available talent pool. Here are some facts suggesting that the talent 
% pool has changed over time. In 1920, the average MLB salary was \$5000 
%(\$60,000 in 2015 dollars). In 2015, the average salary is \$4 million. 
%Before 1947, African-American ball players were excluded from participation 
%in MLB games. Many MLB players and/or potential MLB players served in the 
%armed forces during WWII, the Korean War, the Vietnam War, and the 
%Gulf Wars.  The metric $x_i^D(t)$ takes none of these factors into 
%additional consideration. 
%%%In addition, there is no consensus on how old-time players would perform 
%%%with modern training at the disposal to themselves and their competition 
%%%as a whole. 

%Instead of the detrended metric \eqref{detrended}, one could just as easily 
%construct the detrended metric 
%\begin{equation} \label{inverse}
%  x_i^{D^{\prime}}(t) = x_i(t)\frac{\langle P(t) \rangle}{\overline{P}}. 
%\end{equation}
%which also maintains the merits of dimensionlessness and detrending as they 
%state it. However, the metric \eqref{inverse} reaches the exact opposite 
%conclusions as the metric \eqref{detrended}. The formulation of 
%$x_i^{D^{\prime}}(t)$ rewards prowess from strong eras and punishes prowess 
%from weak eras.  The only difference between the formulations of the two 
%detrending metrics is the placement of $\langle P(t) \rangle$ relative to 
%$\overline{P}$.  The mathematical formulation of PPS detrending is an 
%artifact of the assumptions made.  Detrending using \eqref{inverse} maintains 
%all of the merits of PPS detrending but reaches opposite conclusions with the 
%justification of having the opposite opinion as the authors of PPS.

With all of this being said, we see PPS detrending as more of an inflationary 
metric of relative prowesses and not an era adjustment.  
We agree with Petersen when he says, 
``detrending corresponds to removing the inflationary factor, so we could 
compare two items like the cost of a candy bar in 1920 to the cost of a 
candy bar in 2000.  In this case, we compare Babe Ruth's home runs--the 
ability of someone to get a home run then versus now--and you see Babe 
Ruth actually hit a lot of home runs on this relative basis,'' 
in a BU Today piece written by Rachel Johnson.
%\citep{johnson2011petersen}.  
However, having higher prowess relative to your peers, hitting more runs in 
this case, is not indicative of a player's prowess with respect to peers from 
fundamentally different eras.  Additionally, failing to account for the fact 
that baseball was segregated before 1947 does not make the statistics fairer 
despite the claim that PPS detrending is a way to do as such.



%\vspace{1cm}\noindent {\bf Side quotes}: \\

%That is, according to passionate baseball fan Alexander Petersen.  Using 
%statistical physics theory, he's found a way to compare baseball players over 
%the generations, whether they played in the dead-ball era in the early 1900s 
%or the steroids era beginning in the 1990s \citep{johnson2011petersen}. \\

%``Basically,'' says Petersen, ``detrending corresponds to removing the 
%inflationary factor, so we could compare two items like the cost of a candy 
%bar in 1920 to the cost of a candy bar in 2000. In this case, we compare Babe 
%Ruth's home runs--the ability of someone to get a home run then versus 
%now--and you see Babe Ruth actually hit a lot of home runs on this relative 
%basis.'' Petersen's new statistics compare career longevity, success, and 
%productivity. When comparing the baseball statistics, he was startled to 
%discover a statistical pattern, what he calls a beautiful power-law curve 
%(a type of bell curve), emerging from seemingly random careers 
%\citep{johnson2011petersen}. \\


%``The relative significance of their accomplishments gets reduced because 
%their contemporaries are hitting a lot of home runs as well,'' says 
%Petersen. Players like Babe Ruth, Lou Gehrig, and Ted Williams shoot up the 
%list because they were giants in their own era as well as across the 
%decades.  Petersen says his approach is a way to make the statistics fairer: 
%besides accounting for the possible effects of steroids, detrending allows 
%for changes in equipment, diet, conditioning, even medical procedures like 
%Tommy John surgery (replacing a ligament in the elbow with a tendon to 
%lengthen a pitching career), all of which have changed the game since 
%Ruth's time \citep{johnson2011petersen}. \\



\subsection{Critique of era bridging}



\citet{berry1999eras} claim that their era bridging technique accounts for 
talent discrepancies across eras.  However, they do not explicitly 
parameterize this in their hierarchical models.  They also conclude that 
``globalization has been less pronounced in the MLB (relative to other 
sports), where players are drawn mainly from the United States and other 
countries in the Americas.  Baseball has remained fairly stable within the 
United States, where it has been an important part of the culture for more 
than a century,`` \citep{berry1999eras}.  This rationale ignores the 
segregation of baseball before 1947, increases in the eligible MLB population 
relative to available roster spots, and increases in the average overall 
talent of the eligible MLB population.  They claim that they capture the 
changing pool through use of separate distributions for each decade which 
allows them to study the changing distribution of players in sports 
over time \citep{berry1999eras}.  This type of adjustment is calculated on 
the achievement itself, not the traits of the individuals producing the 
achievement.  Therefore it does not study the characteristics of a 
changing talent pool. We can see this in 
\citet[panel (c) of Figure 7]{berry1999eras}.
In this figure we see that their model predicts that a .300 hitter in 1996 
will have a lower than .300 average for several seasons from 1900-1920. 
This conflicts with the well established notion that the talent of baseball 
players has improved over time.  

We can see that their methodology does not incorporate changing population 
dynamics through examining their \citet[Table 9]{berry1999eras}.  In this 
table they find that 6 of the 10 best hitters for average started their 
career before 1950 and 10 of the 25 best hitters for average started their 
career before 1950.  Their paper came out in 1999 so we recompute the 
chances of these events where the eligible MLB population ends at 1999 
(we approximate this by taking the population counts in the first 12 rows 
of Table 1 and then multiplying the population count in 2000 by 0.9).  
We calculate the chance that one would expect to observe 6 or more 
individuals in a top 10 list who started their career before 1950 as 
1 in 30. 
We calculate the chance that one would expect to observe 10 or more 
individuals in a top 25 list who started their career before 1950 as 
1 in 7.7.  These chances are not as extreme 
as those in Table~\ref{probvalues} but they still correspond to events that 
are highly unlikely.  



% Batting average does not have the same increase over the century. 
% There is a gradual increase in the ability of players
% to hit for average, but the increase is not nearly as dramatic
% as for home runs. The distribution of batting average players
% lends good support to Gould's conjecture. The best players
% are increasing in ability, but the median and 10th percentile
% players are increasing faster over the century. It has gotten
% harder for players of a fixed ability to hit for average. This
% may be due to the increasing ability of pitchers.

\subsection{Critique of our testing procedure}
\label{ourcritique}
Our testing procedure is valid under two assumptions made in 
Section~\ref{sec:Stats}.  A sensitivity analysis is given in 
Section~\ref{sec:Assumptions} that considers how our findings change under 
violations of our assumptions.  The results of the sensitivity analysis are  
largely consistent with our original analysis, with an exception being when 
we weigh populations with respect to a weighting scheme which states that 
the eligible MLB population were those who listed baseball as their favorite 
sport.  





\begin{table}[h!]
\begin{center}
\begin{tabular}{lccccc}
\hline
year & original data & w1 & w2 & w3 & w4 \\
\hline
1880 & 37   
  & 18.5   
  & 22.2   
  & 14.1 
  & 18.1   \\
1890 & 41.7   
  & 20.9   
  & 25.1   
  & 15.9 
  & 20.5   \\
1900 & 46.5   
  & 23.2   
  & 27.9   
  & 17.7 
  & 22.8   \\
1910 & 21.4   
  & 10.7   
  & 12.8   
  & 8.1 
  & 10.5   \\
1920 & 22.3   
  & 11.2   
  & 13.4   
  & 8.5 
  & 10.9   \\ 
  %& & & & & \\   
1930 & 24.8  
  & 12.4  
  & 14.9  
  & 9.4
  & 12.2  \\
1940 & 27.8  
  & 11.1  
  & 11.1  
  & 9.7
  & 10.4  \\
1950 & 29  
  & 11.6  
  & 11.6  
  & 11
  & 11.3  \\
1960 & 46.1  
  & 18.4  
  & 18.4  
  & 15.7
  & 17  \\  
1970 & 40.8  
  & 16.3  
  & 16.3  
  & 11.4
  & 13.9  \\   
  %& & & & & \\ 
1980 & 52.2 
  & 20.9 
  & 20.9 
  & 8.4
  & 14.6 \\
1990 & 57.6 
  & 23.1 
  & 23.1 
  & 9.2
  & 16.1 \\  
2000 & 80.9 
  & 32.4 
  & 32.4 
  & 10.5
  & 21.4 \\
2010 & 96.4 
  & 38.5 
  & 38.5 
  & 11.6
  & 25.1 \\
\hline
\end{tabular}
\end{center}
\caption{Relative talent dilution when considering the weighted eligible MLB 
  population sizes. Numbers are in thousands of people per roster spot. }
\label{dilution2}
\end{table}




\section{Discussion}

Our methodology holds up well when we weight our original data in ways that 
reflect relative disbelief in our assumptions.  Moreover, we expect that the 
conclusions drawn from weighting with respect to our third weighting scheme 
are extremely unlikely.  To see this we investigate talent dilution with 
respect to the four weighted populations as depicted in 
Table~\ref{dilution2}.  From the w3 column of Table~\ref{dilution2}, we see 
that talent was least diluted at and before 1900.  If that were truly the 
case then we would expect less variability and outliers in achievements 
as described by Stephen J. Gould in numerous books, articles, and YouTube 
videos.  The baseball reference leader boards for these years 
indicate the exact opposite. In particular, see the season batting average 
leaders. % in \citet{bref}. 
This suggests that the talent pool in the MLB was diluted during that 
time period.  

PPS does not offer any insight on how their method fairs when their underlying 
assumptions are called into question.  
As argued in Section~\ref{WARcritique}, methodology which compares players 
across eras by examining who stood the farthest from their peers is 
fundamentally flawed.  Era bridging offers a step in the right direction but 
the methodology does not explicitly account for a changing talent pool.


The MLB players from the old eras of baseball receive significant attention 
and praise as a result of their statistical achievements and their mythical 
lore.  We argue that these players are collectively overrepresented in 
rankings of the greatest players in the history of the MLB.  
Our statistical methodology provides overwhelming evidence in 
favor of this argument. We conclude that the superior statistical 
accomplishments achieved by players that started their careers before 1950 
are a reflection of our inability to properly compare talent across eras.  
It is highly unlikely that athletes from such a scarcely populated era of 
available baseball talent could represent top 10 and top 25 lists so 
abundantly. 


%\section*{Acknowledgements}
%This work was partially supported by NIH grants NICHD DP2 HD091799-01.
%I am grateful to
%Jim Albert, 
%Peter M. Aronow, 
%James Burrell, 
%R. Dennis Cook, 
%Forrest W. Crawford, 
%Andrew Depuy, 
%Evan Eck, 
%Kim Eck, 
%Marcus A. Eck, 
%Michael Eck, 
%Phil Eck, 
%Wes Eck, 
%Charles J. Geyer, 
%Adam Maidman, 
%Oliver Om,
%Ken Ressel, 
%Erick Ruuttila, 
%Jesse Ruuttila, 
%Anne Schuh, 
%Bill Schuh, 
%Yushuf Sharker, 
%Ben Sherwood, 
%Stephanna Szotkowski,
%Dootika Vats,
%and
%Brandon Whited 
%for helpful comments and discussions.


%\section*{Appendix A: Mathematical Details}
%We revisit the calculation discussed in the Statistical Evidence section. The chance that 
%an eligible MLB player whose career was largely before 1950 appears in the top 
%10 occurs with probability $p = signif(data[8,3], 3)$ 
%(up to three significant figures).  We use the binomial distribution to 
%calculate the probability of at least 6 eligible MLB players whose career 
%started before 1950 appear in the top 10.  More formally, we are interested in 
%calculating $P(X \geq 6)$ where $X$ is the number of players whose careers 
%started before 1950 appearing in the top 10.  So we have
%$$  
%  P(X \geq 6) = P(X = 6) + \cdots + P(X = 10). 
%$$
%The binomial distribution is used to calculate $P(X = 6)$, $\ldots$, 
%$P(X = 10)$ where
%$$  
%  P(X = x) = {10 \choose x}p^x(1-p)^{10-x}
%$$
%for $x = 6,7,8,9,10$. For example, when $x = 6$ we have
%$$  
%  P(X = 6) = {10 \choose 6}p^6(1-p)^{4}. 
%$$
%Specifying $x = 6$ is equivalent to 6 eligible MLB players whose careers started 
%before 1950 appearing in the top 10.  We do not care about the order 
%that players appear in the top 10.  The number ${10 \choose 6}$ corrects for 
%this. The number ${10 \choose 6}$ is the number of possible orderings of 6 
%``successes" and 4 ``failures" that can occur.  In our context a success is a 
%MLB player whose career began before 1950 and a failure is an 
%MLB player whose career began after 1950.  Any particular ordering of the top 
%10 with 6 MLB players whose careers started before 1950 appearing 
%in the list occurs with probability $p^6(1-p)^4$. 
%Therefore the probability that 6 MLB players who started their careers  
%before 1950 appear in the top 10 is given by
%$$  
%  P(X = 6) = {10 \choose 6}p^6(1-p)^{4} 
%    = choose(10, 6) \times p^6 \times (1 - p)^4
%    = dbinom(6, size = 10, p = p) . 
%$$
%R statistical software was used as a calculator throughout the Statistical 
%Evidence section.  Similar calculations to this one are made to calculate 
%all of the probabilities that appear in Tables~\ref{probvalues} and 
%\ref{probvalues.weights}.


\section*{Appendix: Conservative Weighting Schemes}
The weights discussed in Section~\ref{sec:Assumptions} are given in the table 
below.  These weights reflect Gallup polling data 
%\citep*{moore, carlson, gallup} 
on the subject of fan interest in baseball. 
Polling data does not exist for all time periods.  
Specifically, there are no records before 1940. The first two weighting 
schemes reflect overall fan interest in baseball. It is apparent from the 
Gallup dataset that the proportion of Americans sampled who are baseball fans 
has been around 40\% from 1940 on. The first weighting scheme (w1) places a 
0.50 weight on fan interest with the assumption that more Americans from those 
time periods favored, and played, baseball due to lack of competition from 
other sports. The second weighting scheme (w2) places even stronger weights 
on pre-1940 fan interest.  The third weighting scheme (w3) 
reflects the proportion of Americans sampled that list baseball as their 
favorite sport.  Again, we place a higher weight for pre-1940 fan interest.  
The fourth weighting scheme (w4) is the average of w2 and w3.  

All four weighting schemes serve as an educated proxy for the eligible MLB 
population thought to strive towards a career in professional baseball.  The 
weighting schemes w2, w3, and w4 are especially conservative.  We do not 
expect fan interest in pre-1940s baseball to be as high as our weighting 
schemes due to lack of media exposure and salary compensation.  
The appropriateness of w3 is particularly questionable since individuals play 
baseball even if it is not the individual's favorite sport.  It should be noted 
that these weights are obtained from survey data from the United States 
only and are applied, at equal value, to the other countries. Therefore our
weighting schemes are biased but rich survey data on the topic of fan 
interest in baseball is unavailable from from the other countries. 


% latex table generated in R 3.4.4 by xtable 1.8-3 package
% Tue May 28 18:40:56 2019
\begin{table}[ht]
\centering
\begin{tabular}{rrrrrr}
  \hline
 & year & w1 & w2 & w3 & w4 \\ 
  \hline
1 & 1880 & 0.50 & 0.60 & 0.38 & 0.49 \\ 
  2 & 1890 & 0.50 & 0.60 & 0.38 & 0.49 \\ 
  3 & 1900 & 0.50 & 0.60 & 0.38 & 0.49 \\ 
  4 & 1910 & 0.50 & 0.60 & 0.38 & 0.49 \\ 
  5 & 1920 & 0.50 & 0.60 & 0.38 & 0.49 \\ 
  6 & 1930 & 0.50 & 0.60 & 0.38 & 0.49 \\ 
  7 & 1940 & 0.40 & 0.40 & 0.35 & 0.38 \\ 
  8 & 1950 & 0.40 & 0.40 & 0.38 & 0.39 \\ 
  9 & 1960 & 0.40 & 0.40 & 0.34 & 0.37 \\ 
  10 & 1970 & 0.40 & 0.40 & 0.28 & 0.34 \\ 
  11 & 1980 & 0.40 & 0.40 & 0.16 & 0.28 \\ 
  12 & 1990 & 0.40 & 0.40 & 0.16 & 0.28 \\ 
  13 & 2000 & 0.40 & 0.40 & 0.13 & 0.27 \\ 
  14 & 2010 & 0.40 & 0.40 & 0.12 & 0.26 \\ 
  15 & 2015 & 0.40 & 0.40 & 0.10 & 0.25 \\ 
   \hline
\end{tabular}
\caption{Weighting schemes.} 
\end{table}



\begin{FlushLeft}
\begin{thebibliography}{00}
\singlespacing

%\bibitem[Albert(2002)Albert]{albert2002course}
%Albert, Jim (2002),
%\newblock ``A Baseball Statistics Course,"
%\newblock \emph{Journal of Statistics Education}, {\bf 10}, 2. 
%\newblock \url{ww2.amstat.org/publications/jse/v10n2/albert.html}

%\bibitem[Armour(2016)Armour]{armour2016integration}
%Armour, M. (2016),
%\newblock ``Baseball Integration, 1947--1986,'' 
%\newblock {\em Society for American Baseball Research}, 
%\newblock \url{https://sabr.org/bioproj/topic/integration-1947-1986}

%\bibitem[Arthur(2016)Arthur]{arthur2016bullpen}
%Arthur, R. (2016),
%\newblock ``In Baseball, October Is Reliever Season,''
%\newblock FiveThirtyEight.
%\newblock \url{https://fivethirtyeight.com/features/in-baseball-october-is-reliever-season/}

%\bibitem[Baseball-Reference(2018)Baseball-Reference]{bref}
%Baseball-Reference.com. (2018). 
%\newblock \url{http://www.baseball-reference.com}

%\bibitem[Badenhausen(2016)Badenhausen]{badenhausen2016salary}
%Badenhausen, K. (2016),
%\newblock ``Average Baseball Salary Up 20,700\% Since First CBA in 1968,''
%\newblock Forbes. 
%\newblock \url{https://www.forbes.com/sites/kurtbadenhausen/2016/04/07/average-baseball-salary-up-20700-since-first-cba-in-1968/#6b05f04b3e48}

\bibitem[Berry et al.(1999)Berry et al.]{berry1999eras}
Berry, S.~M., Reese, C.~S., and Larkey, P.~D. (1999),
\newblock ``Bridging Different Eras in Sports,''
\newblock \emph{Journal of the American Statistical Association}, {\bf 94}, 447, 661--676.

%\bibitem[Carlson(2001)Carlson]{carlson}
%Carlson, D.~K. (2001),
%\newblock ``Although Fewer Americans Watch Baseball, Half Say They are Fans,''
%\newblock Gallup.
%\newblock \url{http://www.gallup.com/poll/4624/Although-Fewer-Americans-Watch-Baseball-Half-Say-They-Fans.aspx?g_source=baseball%20interest&g_medium=search&g_campaign=tiles}

%\bibitem[Castrovince(2016)Castrovince]{castrovince2016velocity}
%Castrovince, A. (2016),
%\newblock ``Speed trap: How velocity has changed baseball,''
%\newblock ESPN. 
%\newblock \url{https://www.mlb.com/news/increase-in-hard-throwers-is-changing-mlb/c-170046614}

%\bibitem[Cenus(2018)Census]{census}
%Census (2018). 
%\newblock \url{https://www.census.gov/}

%\bibitem[Cenus(1967)Census]{census1967}
%Census (1967),
%\newblock ``Household Income in 1967 and selected social and economic 
%characteristics of households.''
%\newblock \url{https://www2.census.gov/prod2/popscan/p60-062.pdf}

%\bibitem[CIA World Factbook(2018)CIA World Factbook]{CIA}
%CIA World Factbook (2018). 
%\newblock \url{https://www.cia.gov/library/publications/the-world-factbook/geos/print_tw.html}

%\bibitem[Doran(2015)Doran]{doran2015velocity}
%Doran, N. (2015),
%\newblock ``Running Down the Velocity Upswing.''
%\newblock \url{https://redlegnation.com/2015/02/19/running-down-the-velocity-upswing/}

%\bibitem[ESPN(2015)ESPN]{ESPN}
%ESPN (2015),
%\newblock ``ESPN's Hall of 100.'' 
%\newblock \url{http://www.espn.com/mlb/feature/video/_/id/8652210/espn-hall-100-ranking-all-greatest-mlb-players}

%\bibitem[ESPN(2015)ESPN]{ESPN2015methodology}
%ESPN (2015), 
%\newblock ``Hall of 100 methodology.''
%\newblock \url{http://www.espn.com/mlb/story/_/id/12127337/hall-100-methodology-mlb}

%\bibitem[Fangraphs(2018)Fangraphs]{fangraphs}
%Fangraphs. (2018). 
%\newblock \url{http://www.fangraphs.com}

%\bibitem[Gallup(2016)Gallup]{gallup}
%Gallup. (2016). 
%\newblock \url{http://www.gallup.com/poll/4735/sports.aspx}

%\bibitem[Gould(1996)Gould]{gould1996fullhouse}
%Gould, S.~J. (1996),
%\newblock ``Full House: The spread of excellence from Plato to Darwin,''
%\newblock New York: Three Rivers.

%\bibitem[Haupert(2016)Haupert]{haupert2016salary}
%Haupert, M. (2016),
%\newblock ``MLB's annual salary leaders since 1874,''
%\newblock {\em Society for American Baseball Research}. 
%\newblock \url{https://sabr.org/research/mlbs-annual-salary-leaders-1874-2012}

%\bibitem[IRS(2018)IRS]{IRS}
%Internal Revenue Service. (2018). 
%\newblock \url{https://www.irs.gov/}

%\bibitem[IRS(1920)IRS]{IRS1920}
%Internal Revenue Service. (1920). 
%\newblock \url{https://www.irs.gov/pub/irs-soi/20soirepar.pdf}

%\bibitem[Johnson(2011)Johnson]{johnson2011petersen}
%Johnson, R. (2011),
%\newblock ``Baseball Greats Reranked.''
%\newblock BU Today. 
%\newblock \url{http://www.bu.edu/today/2011/baseball-greats-reranked/}

%\bibitem[Kahrl(2016)Kahrl]{kahrl2016wagner}
%Kahrl, C. (2016),
%\newblock ``Time to reconsider Honus Wagner's and Lou Gehrig's greatness?''
%\newblock ESPN. 
%\newblock \url{http://www.espn.com/mlb/story/_/page/mlbrank100_wagnergehrig/mlbrank-exploring-greatness-honus-wagner-lou-gehrig}

%\bibitem[Keh(2013)Keh]{keh2013offseason}
%Keh, A. (2013),
%\newblock ``Chances Are Extra Work in Off-Season Involves Baseball, Not a Second Job,''
%\newblock New York Times. 
%\newblock \url{https://www.nytimes.com/2013/02/24/sports/baseball/for-baseball-players-finding-work-in-off-season-is-no-longer-a-necessity.html}

%\bibitem[Koppelt(2007)Koppelt]{NBA}
%Koppelt, L. (2007),
%\newblock ``The NBA -- 1946: A New League,''
%\newblock NBA.com. 
%\newblock \url{http://www.nba.com/heritageweek2007/newleague_071207.html}

%\bibitem[Lewis(2004)Lewis]{lewis2004moneyball}
%Lewis, M. (2004),
%\newblock ``Moneyball: The art of winning an unfair game,''
%\newblock WW Norton \& Company

%\bibitem[Lor{\'e}(2017)Lor{\'e}]{Brazil}
%Lor{\'e}, Michael (2017),
%\newblock ``Baseball is Gaining Popularity in Brazil,''
%\newblock Culture Trip. 
%\url{https://theculturetrip.com/south-america/brazil/articles/baseball-is-gaining-popularity-in-brazil/}

%\bibitem[Milan and Tran(2004)Milan and Tran]{milan2004blacks}
%Milan, A. and Tran, K. (2004),
%\newblock ``Blacks in Canada: a long history,'' 
%\newblock {\em Canadian Social Trends Spring}, 2--7

%\bibitem[Moore and Carroll(2002)Moore and Carroll]{moore}
%Moore, D.~W. and Carroll, J. (2002),
%\newblock ``Baseball Fan Numbers Steady, But Decline May Be Pending,''
%\newblock Gallup.
%\newblock \url{http://www.gallup.com/poll/6745/Baseball-Fan-Numbers-Steady-Decline-May-Pending.aspx?g_source=baseball%20interest&g_medium=search&g_campaign=tiles}

%\bibitem[NFL(2017)NFL]{NFL}
%\newblock Official 2017 National Football League Record \& Factbook. (2017). 
%\newblock \url{https://static.nfl.com/static/content/public/photo/2017/08/22/0ap3000000833458.pdf#page=357}

\bibitem[Petersen et al.(2011)Petersen, Penner, and Stanley]{petersen}
Petersen, A.~M., Penner, O., Stanley, H.~E. (2011),
\newblock ``Methods for detrending success metrics to account
  for inflationary and deflationary factors,''
\newblock \emph{The European Physical Journal B}, \textbf{79}, 67--78.

%\bibitem[Radcliffe(2018)Radcliffe]{radcliffe2018salary}
%Radcliffe, B. (2018),
%\newblock ``Baseball Greats Who Were Paid Like Benchwarmers,''
%\newblock Investopedia. 
%\newblock \url{https://www.investopedia.com/financial-edge/0510/baseball-greats-who-were-paid-like-benchwarmers.aspx}

%\bibitem[Ranker(2018)Ranker]{ranker2018greatest}
%Ranker (2018),
%\newblock ``The Greatest Baseball Players of All Time.''
%\newblock \url{https://www.ranker.com/crowdranked-list/the-greatest-baseball-players-of-all-time}

%\bibitem[Reichler(1985)Reichler]{reichler1985encyclopedia}
%Reichler, J.~L. (1985),
%\newblock ``The baseball encyclopedia: The complete and official record of major league baseball,''
%\newblock Macmillan Pub. Co.

%\bibitem[Schoenfield(2012)Schoenfield]{schoenfield2012war}
%Schoenfield, D. (2012),
%\newblock ``What we talk about when we talk about WAR,''
%\newblock ESPN. 
%\newblock \url{http://www.espn.com/blog/sweetspot/post/_/id/27050/what-we-talk-about-when-we-talk-about-war}

\bibitem[Schmidt and Berri(2005)Schmidt and Berri]{schmidt2005concentration}
Schmidt, M.~B. and Berri, D.~J. (2005),
\newblock ``Concentration of Playing Talent: Evolution in Major League Baseball,''
\newblock \emph{Journal of Sports Economics}, {\bf 6}, 412--419.

%\bibitem[Shaikin(2016)Shaikin]{shaikin2016salary}
%Shaikin, B. (2016),
%\newblock ``A look at how Major League Baseball salaries have grown more than 20,000\% the last 50 years,''
%\newblock Los Angeles Times. 
%\newblock \url{http://www.latimes.com/sports/mlb/la-sp-mlb-salaries-chart-20160329-story.html}

%\bibitem[Statistics Canada(2018)Statistics Canada]{statcan}
%Statistics Canada (2018). 
%\newblock \url{https://www.statcan.gc.ca/eng/start}

%\bibitem[Thorn(2014)Thorn]{thorn2014pitching}
%Thorn, J. (2014),
%\newblock ``Pitching: Evolution and Revolution,''
%\newblock \url{https://ourgame.mlblogs.com/pitching-evolution-and-revolution-efd3a5ebaa83}

%\bibitem[United Nations(2011)United Nations]{UN}
%United Nations, Department of Economic and Social Affairs, Population Division (2011).
%\newblock World Population Prospects: The 2010 Revision, CD-ROM Edition.

\end{thebibliography}
\end{FlushLeft}




\end{document}
